\documentclass[a4paper,11pt]{article}
\usepackage[american]{babel}
\usepackage{hyperref}
\usepackage{geometry}
\usepackage{sectsty}
\usepackage[table,svgnames]{xcolor}
\usepackage{graphicx}
\usepackage{listings}
\usepackage[T1]{fontenc}
\usepackage[utf8]{inputenc}
\usepackage{mathptmx}

\newcommand{\authortitle}{Priv.-Doz.~Dr.~Ing.}
\newcommand{\authorname}{Martin Lambers}
\newcommand{\course}{Rendering}
\newcommand{\tutorial}{Tutorial 1}

\hypersetup{
  colorlinks = true,
  urlcolor = blue,
  citecolor = blue,
  pdfauthor = {\authorname},
  pdftitle = {\course{} \tutorial},
  pdfsubject = {\course Tutorial},
  pdfpagemode = UseNone
}

\geometry{
  body={6.6in, 8.5in},
  left=1.0in,
  top=1.25in
}

% Custom section fonts
\sectionfont{\rmfamily\mdseries\Large}
\subsectionfont{\rmfamily\mdseries\large}
\subsubsectionfont{\rmfamily\mdseries\normalsize}
\paragraphfont{\rmfamily\mdseries\itshape\normalsize}

% Don't indent paragraphs; leave space instead
\setlength\parindent{0em}
\setlength\parskip{.5\baselineskip plus .5\baselineskip}

% In-text code sample
\newcommand{\code}[1]{\texttt{#1}}
% Larger code examples with lstlisting
\definecolor{cgblue}{rgb}{0.2,0.2,0.7}
\lstloadlanguages{C}
\lstdefinelanguage[OpenGL]{C}[ANSI]{C}{%
    morekeywords={bool,bvec2,bvec3,bvec4,ivec2,ivec3,ivec4,vec2,vec3,vec4}
}
\lstset{%
    language=[OpenGL]C,
    basicstyle=\footnotesize\ttfamily,%
    keywordstyle=\color{cgblue},%
    directivestyle=\color{cgblue},%
    identifierstyle=,%
    commentstyle=\color{cgblue!50!white}\itshape,%
    stringstyle=\color{cgblue!80!white},%
    numbers=none,%
    numberstyle=\tiny,%
    extendedchars=true,%
    showspaces=false,%
    showstringspaces=false,%
    showtabs=false,%
    breaklines=false,%
    frame=single,%
    frameround=tttt,%
    backgroundcolor=\color{white},%
    literate={\\\%}{\%}1,
    escapechar=\%
}


\begin{document}

\thispagestyle{empty}

\LARGE

\centerline{\course{} \tutorial}

\vspace{1ex}

\normalsize

\centerline{\authortitle{} \authorname}

%\vspace{.5\baselineskip}

\section*{Assigment 1}

In the lecture we discussed an initial implementation of a brute force path
tracer.

\begin{enumerate}
\item Get the source code from the course material.
\item Compile and link the program on the system of your choice.
    \begin{itemize}
    \item The program is written in C++ version 2020
    \item The build system is \href{https://cmake.org/}{CMake}
    \item No external libraries are required
    \end{itemize}
\item Run the program.
\end{enumerate}

The program writes an output image in
\href{http://www.pauldebevec.com/Research/HDR/PFM/}{PFM format}. This format allows to
store floating point data for the RGB channels so that we do not need to deal
with conversion to integers, and it is extremely simple so that we do not need
to use an external library. \href{https://www.gimp.org/}{The Gimp} can open PFM
images. The \href{https://marlam.de/qv}{qv} viewer offers some inspection
capabilities (statistics, histogram, pixel values, \ldots{}). Note that Adobe Photoshop imports PFM images upside down; you can work
around this bug by flipping the image vertically.

~

\fbox{\parbox{\textwidth}{%
\begin{center}
Keep track of all changes you make to the program during the tutorial,\\
and be sure to verify their correctness!\\
Each modification will be required in future tutorials,\\
and you will probably 
want to use your modified path tracer as a basis for the project task!\\~\\
There will be no downloadable sample solutions!
\end{center}
}}

\section*{Assigment 2}

Implement Anti-Aliasing for the brute force path tracer, as discussed in the
lecture.

\end{document}
