\documentclass[a4paper,11pt]{article}
\usepackage[american]{babel}
\usepackage{hyperref}
\usepackage{geometry}
\usepackage{sectsty}
\usepackage[table,svgnames]{xcolor}
\usepackage{graphicx}
\usepackage{listings}
\usepackage[T1]{fontenc}
\usepackage[utf8]{inputenc}
\usepackage{mathptmx}

\newcommand{\authortitle}{Priv.-Doz.~Dr.~Ing.}
\newcommand{\authorname}{Martin Lambers}
\newcommand{\course}{Rendering}
\newcommand{\tutorial}{Tutorial 4}

\hypersetup{
  colorlinks = true,
  urlcolor = blue,
  citecolor = blue,
  pdfauthor = {\authorname},
  pdftitle = {\course{} \tutorial},
  pdfsubject = {\course Tutorial},
  pdfpagemode = UseNone
}

\geometry{
  body={6.6in, 8.5in},
  left=1.0in,
  top=1.25in
}

% Custom section fonts
\sectionfont{\rmfamily\mdseries\Large}
\subsectionfont{\rmfamily\mdseries\large}
\subsubsectionfont{\rmfamily\mdseries\normalsize}
\paragraphfont{\rmfamily\mdseries\itshape\normalsize}

% Don't indent paragraphs; leave space instead
\setlength\parindent{0em}
\setlength\parskip{.5\baselineskip plus .5\baselineskip}

% In-text code sample
\newcommand{\code}[1]{\texttt{#1}}
% Larger code examples with lstlisting
\definecolor{cgblue}{rgb}{0.2,0.2,0.7}
\lstloadlanguages{C}
\lstdefinelanguage[OpenGL]{C}[ANSI]{C}{%
    morekeywords={bool,bvec2,bvec3,bvec4,ivec2,ivec3,ivec4,vec2,vec3,vec4}
}
\lstset{%
    language=[OpenGL]C,
    basicstyle=\footnotesize\ttfamily,%
    keywordstyle=\color{cgblue},%
    directivestyle=\color{cgblue},%
    identifierstyle=,%
    commentstyle=\color{cgblue!50!white}\itshape,%
    stringstyle=\color{cgblue!80!white},%
    numbers=none,%
    numberstyle=\tiny,%
    extendedchars=true,%
    showspaces=false,%
    showstringspaces=false,%
    showtabs=false,%
    breaklines=false,%
    frame=single,%
    frameround=tttt,%
    backgroundcolor=\color{white},%
    literate={\\\%}{\%}1,
    escapechar=\%
}


\begin{document}

\thispagestyle{empty}

\LARGE

\centerline{\course{} \tutorial}

\vspace{1ex}

\normalsize

\centerline{\authortitle{} \authorname}

%\vspace{.5\baselineskip}

\section*{Assigment 1}

Implement the \code{SurfaceTriangle} class, in particular the \code{hit()}
function based on the Möller-Trumbore algorithm, as discussed in the
lecture.

Code that implements a triangle mesh and its import from OBJ and MTL files is given in
the tutorial material. Additionally, basic tone mapping functionality is
included.

A skeleton implementation of \code{SurfaceTriangle} is given in the
\code{surface\_triangle.hpp} file. Use this as a starting point to avoid
some pitfalls.

With a correct implementation, the pathtracer program will render the
\code{CornellBox-Original} scene and generate a PPM image file with standard
dynamic range in addition to the PFM file that contains the original HDR image.


\end{document}
